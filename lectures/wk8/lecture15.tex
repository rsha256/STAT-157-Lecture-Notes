\section{Thursday, March 7th}
\subsection{Method of Types}
\begin{itemize}
    \item $X\in \cX, |\cX|<\infty$
    \item $X^{(n)} = \{X_i\}_{i=1}^n, X\simiid p_0$
\end{itemize}

\subsection{Empirical Distribution}
$X^{n} = x^{n}$ is 
\begin{itemize}
    \item Draw $J \sim \Uniform[1, n]$
    \item Y = $X_j$
\end{itemize}
$P(Y = y) = \frac{\# \text{ of } \st x_j = y}{n}$

\begin{equation}
    \cP^{(n)} = \{\text{frequency of the outcome } y \text{ among } \{x_{i}\}_{i=1}^n\}
\end{equation}

\subsection{Type}
\textbf{Definition:} A type $P$ is a probability distribution on $|\cX|$ outcomes w/ rational prob., denominator is $n$.

\subsubsection{Type Class}
The \underline{type-class} $T(P)$, 
\begin{equation}
    T(P) = \{ x^{(n)} \text{ s.t. empirical dist } P_{X^{(n}}(\cdot)=P\}
\end{equation}
\begin{enumerate}
    \item The size of $T(P)$ is combinatorial, \# distinct rearrangements of $x^{(n)}$ of $T(P_{x^{(n)}})$
    \item all draws $x^{(n)}\in T(P)$ are equiprobable
\end{enumerate}

Looking at draws (type class) of length 6, when the event can have 3 outcomes, we view them as points in the simplex (space of possible distributions) as opposed to bar charts.

\subsection{Theorem 11.1.1}
As stated in T\&C, 
The number of types with denominator $n$ is bounded by
$$
\left|\mathcal{P}_n\right| \leq(n+1)^{|\mathcal{X}|}.
$$

\subsection{Theorem 11.1.2}
As stated in T\&C, If $X_1, X_2, \ldots, X_n$ are drawn i.i.d. according to $Q(x)$, the probability of $x^n$ depends only on its type and is given by
$$
Q^n\left(x^n\right)=2^{-n\left(H\left(P_x n\right)+D\left(P_x n || Q\right)\right)}.
$$

\subsubsection{Multiplicity vs Fitting to Data Generating Distribution}
Multiplicity: $H_d(P_{x^{(n)}})$.

Fitting to Data Generating Distribution $p_0$: $D_d(P_{x^{(n)}} || p_0)$

\subsection{Theorem 11.1.3: Size of a type class T(P)}
For any type $P \in \mathcal{P}_n$,
$$
\underbrace{\frac{1}{(n+1)^{|\mathcal{X}|}}}_{\text{Polynomial}} d^{n H(P)} \leq|T(P)| \leq d^{n H(P)} .
$$

This goes beyond excess surprise and gives us something to bound any probability distribution by.

\subsection{Theorem 11.1.4: Prob. of a type class T(P)}
For any $P \in \mathcal{P}_n$ and any distribution $Q$, the probability of the type class $T(P)$ under $Q^n$ is $2^{-n D(P \| Q)}$ to first order in the exponent. More precisely,
$$
\frac{1}{(n+1)^{|\mathcal{X}|}} 2^{-n D(P \| Q)} \leq Q^n(T(P)) \leq 2^{-n D(P \| Q)} .
$$

\subsection{Key Equation}
$$
D(P \| Q) \stackrel{\tiny{n\to\infty}}\simeq =-\frac1n \log_d(P(X^{(n)} \in T(P)).
$$
where $\simeq$ means asymptotically equal. This is used in several proofs, including that of CLT.
